\documentclass[12pt]{article}
\usepackage[spanish,activeacute]{babel}
\usepackage[utf8]{inputenc}
\usepackage[spanish]{babel}
\spanishdecimal{.}
\usepackage{amsmath}
\usepackage{amsfonts}
\usepackage{amssymb}
\usepackage{graphicx}
\usepackage[usenames,dvipsnames]{pstricks}

\usepackage{siunitx}
\usepackage{wasysym}
\topmargin -2.5 cm
\oddsidemargin -1.5 cm
\textheight 23.5 cm
\textwidth 18.5cm

\title{Satélite Orbital}
\author{Juan Felipe Agudelo Arboleda \\c.c:1037626275}
\begin{document}
\maketitle


El texto está disponible bajo la Licencia Creative Commons Atribución Compartir Igual 3.0
\begin{enumerate} 

\item Un satélite de masa $m$ se encuentra en una órbita eliptica alrededor de la Tierra. En el apogeo su altitud es $h$ y en el perigeo $h'$. 
  \begin{minipage}{0.6\linewidth}
    \begin{enumerate}
    \item Energía del satélite si $h=1100 km$ y $h'=4100 km$.
      \label{item:JFa}
    \item Halle la energía minima de lanzamiento para los mismos valores anteriores.
      \label{item:JFb}
      \item Momento angular para los mismos valores anteriores.
      \label{item:JFc}    
     \item Halle la velocidad en el apogeo y en el perigeo.
      \label{item:JFd}
    \end{enumerate}
  \end{minipage}
  
  \textbf{Solución}
  \begin{itemize}  
  
  \item[\ref{item:JFa})]
  Primeramente debemos tener presente que como la masa del satélite es muchisimo menor que la de la Tierra es totalmente válido asumir
\begin{equation} 
m<<M\rightarrow\mu=m\nonumber
\end{equation}

Plasmemos en el siguiente gráfico la situacion planteada:\\

  \begin{eqnarray}
  Re&=&6400 km=6400000 m\nonumber\\ 
  A&=&(1100+4100+(2\times6400)) km\nonumber\\
  A&=&18000  km = 1.8\times10^4 km = 1.8\times10^7 m\\
   \nonumber
 \end{eqnarray}
 
 Para calcular la energía usamos la ecuación:
\begin{equation}
A=\frac{C}{-E}\nonumber
\end{equation}
donde
\begin{equation}
C=G M m\nonumber
\end{equation} 
Con un analisis de fuerzas podemos dejar $C$ en una expresión mas simple:\\
\begin{center}


\begin{eqnarray}
+\downarrow \sum f_y \Longrightarrow f(r)&=&m g\nonumber\\
\frac{G M m}{Re^2}&=&m g\nonumber\\
\frac{G M}{Re^2}&=&g\nonumber\\
G M&=&g Re^2\\
\nonumber
\end{eqnarray}    

\end{center}
Reemplazando la igualdad (2) en la ecuación de $C$\\
\begin{eqnarray}
C&=&G M m\nonumber\\
C&=&m g Re^2\nonumber\\
C&=&(2000 kg)(9.8\frac{m}{s^2})(6400000 m)^2\nonumber\\
C&=&8.028\times 10^{17} J.m
\end{eqnarray}
Ahora que conocemos los valores de $A$ y $C$ podemos usar la ecuación propuesta al inicio:\\
\begin{eqnarray}
E&=&-\frac{C}{A}\nonumber\\
E&=&-\frac{8.028\times 10^{17} J.m}{1.8\times10^7 m}\nonumber\\
E&=&-4.46\times10^{10} J
\end{eqnarray}
  
  \item[\ref{item:JFb})] 
  La energía minima de lanzamiento está dada por:
 \begin{equation} 
 E_{min}=E-E_i\nonumber
 \end{equation}
 Para la energía inicial podemos usar el momento antes del lanzamiento, es decir, cuando la energía solo es energía gravitacional potencial.\\
 \begin{eqnarray}
 E_i&=&-\frac{G M m}{Re}\nonumber\\
 E_i&=&-\frac{C}{Re}\nonumber\\
 E_i&=&-1.25\times10^{11} J\\
 \nonumber
 \end{eqnarray}
 En consecuencia la energía minima de lanzamiento es:\
 \begin{eqnarray}
 E_{min}&=&E-E_i\nonumber\\
 E_{min}&=&-4.46\times10^{10} J + 1.25\times10^{11} J\nonumber\\
 E_{min}&=&8.04\times10^{10} J\\
 \nonumber
\end{eqnarray}
  
  
  
\item[\ref{item:JFc})]
El momento angular lo podemos calcular con la siguiente ecuación:\\
\begin{equation}
\varepsilon^2=1+\frac{2 E L^2}{m C^2}
\end{equation}
Pero aún no conocemos la excentricidad, así que hay que calcularla primero:\\
\begin{eqnarray}
\varepsilon&=&\frac{r_{max}-r_{min}}{A}\nonumber\\
\varepsilon&=&\frac{4100 km - 1100 km}{1.8\times10^4 km}\nonumber\\
\varepsilon&=&\frac{3000 km}{1.8\times10^4 km}\nonumber\\
\varepsilon&=&0.166\\
\nonumber
\end{eqnarray}
Ahora despejando L de la ecuación de excentricidad:\\
\begin{eqnarray}
\varepsilon^2-1&=&\frac{2 E L^2}{m C^2}\nonumber\\
\sqrt{\frac{(\varepsilon^2-1)(m C^2)}{2 E}}&=&L\nonumber\\
\sqrt{\frac{(0.027-1)(2000 kg (8.028\times10^{17} J.m)^2)}{2(-4.46\times10^{10} J)}}&=&L\nonumber\\
1.18\times10^{14} \frac{kg.m^2}{s}&=&L\\
\nonumber
\end{eqnarray} 

  
\item[\ref{item:JFd})] 
En primer lugar calculamos la velocidad en el perigeo con la ecuacion:\\

\begin{eqnarray}
E&=&\frac{m V^2}{2}-\frac{C}{r_{min}}\nonumber\\
E+\frac{C}{r_{min}}&=&\frac{m V^2}{2}\nonumber\\
\sqrt{\frac{2}{m} \left(E+\frac{C}{r_{min}}\right)}&=&V\\
\nonumber
\end{eqnarray}

\begin{eqnarray}
 r_{min}&=&6400 km + 1100 km = 7500 km = 7500000 m\nonumber\\ 
 V_p&=&\sqrt{\frac{2}{2000 kg}\left(-4.46\times10^{10} J + \frac{8.028\times10^{17} J.m}{7500000 m}\right)}\nonumber\\
 V_p&=&7901.9 \frac{m}{s}\\
 \nonumber
\end{eqnarray}

Finalmente calculamos la velocidad en el apogeo, con una ecuación relativamente mas sencilla que la usada para el perigeo.\\
\begin{eqnarray}
V_a&=&\frac{V_p r_{min}}{r_{max}}\nonumber\\
V_a&=&\frac{7500000 m \times 7901.9 \frac{m}{s}}{10500000 m}\nonumber\\
V_a&=&5644.2 \frac{m}{s}\\
\nonumber
\end{eqnarray}
 
\end{itemize}

\end{enumerate}
\end{document}
